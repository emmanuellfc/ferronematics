\documentclass[oneside,english]{amsart}
\usepackage{mathpazo}
\usepackage[T1]{fontenc}
\usepackage[latin9]{inputenc}
\usepackage{geometry}
\geometry{verbose,tmargin=3.5cm,bmargin=3.5cm,lmargin=3cm,rmargin=3cm}
\setlength{\parskip}{\smallskipamount}
\setlength{\parindent}{0pt}
\usepackage{amstext}
\usepackage{amsthm}
\usepackage{setspace}
\usepackage{listings}

\onehalfspacing

\makeatletter

%%%%%%%%%%%%%%%%%%%%%%%%%%%%%% LyX specific LaTeX commands.
%% Because html converters don't know tabularnewline
\providecommand{\tabularnewline}{\\}

%%%%%%%%%%%%%%%%%%%%%%%%%%%%%% Textclass specific LaTeX commands.
\numberwithin{equation}{section}
\numberwithin{figure}{section}

\makeatother

\usepackage{babel}
\begin{document}
\title{Ferronematics: Technical Note}

\maketitle
Key findings of the model:
\begin{itemize}
\item The control parameter is the temperature, and they observe a transition
form CV-type topologies to L-type topologies as function of this parameter.
In particular, as the temperature decreased, the droplets grew, and
the polar topology transitioned from the CV-type to the L-type structure.
\item On the other hand, by comparing droplets\textquoteright{} free energy
landscapes, they were able to prove that elastic anisotropy minimally
affects polarization topology, which implies that a one single-elastic-approximation
is enough for describing the transition.
\item So, what are the core elements for the transition?
\end{itemize}
The free energy is taken from \cite{key-1}

\begin{align*}
f & =\frac{K_{11}}{2}\left(\nabla\cdot\mathbf{N}\right)^{2}+\frac{K_{22}}{2}\left(\mathbf{N}\cdot\nabla\times\mathbf{N}\right)^{2}+\frac{K_{33}}{2}\left(\mathbf{N}\times\nabla\times\mathbf{N}\right)^{2}\\
 & +\frac{\tau_{1}}{2}\mathbf{P}^{2}+\frac{\tau_{2}}{2}\mathbf{P}^{4}+\frac{h}{2}\left(\nabla\mathbf{P}\right)^{2}-\gamma\left(\nabla\cdot\mathbf{N}\right)\mathbf{N}\cdot\mathbf{P}-\frac{1}{2}\mathbf{P}\cdot\mathbf{E}_{d}.
\end{align*}
On the other hand, the surface anchoring is given by 
\[
f_{s}=\frac{1}{2}W_{Q}\left[1-\left(\mathbf{n}\cdot\mathbf{R}_{0}\right)^{2}\right]-W_{P}\left[\mathbf{n}\cdot\mathbf{R}_{0}-1\right].
\]
Now, considering a linear coupling between the nematic order parameter
and the polarization, i.e., $\mathbf{P}=P_{0}\mathbf{N}$, where $\mathbf{N}=s\mathbf{n}$,
we have the bulk free energy, without taking into consideration the
depolarization effect, is given by 
\begin{align*}
f= & \frac{K_{11}}{2}\left(\nabla\cdot\mathbf{N}\right)^{2}+\frac{K_{22}}{2}\left(\mathbf{N}\cdot\nabla\times\mathbf{N}\right)^{2}+\frac{K_{33}}{2}\left(\mathbf{N}\times\nabla\times\mathbf{N}\right)^{2}\\
 & +\frac{\tau_{1}}{2}P_{0}^{2}\mathbf{N}^{2}+\frac{\tau_{2}}{2}P_{0}^{4}\mathbf{N}^{4}+\frac{h}{2}P_{0}^{2}\left(\nabla\mathbf{N}\right)^{2}-\gamma P_{0}\left(\nabla\cdot\mathbf{N}\right)\mathbf{N}\cdot\mathbf{N}.
\end{align*}

\begin{align*}
f= & \frac{K_{11}}{2}\frac{1}{\xi^{2}}\left(\nabla\cdot\mathbf{N}\right)^{2}+\frac{K_{22}}{2}\frac{1}{\xi^{2}}\left(\mathbf{N}\cdot\nabla\times\mathbf{N}\right)^{2}+\frac{K_{33}}{2}\frac{1}{\xi^{2}}\left(\mathbf{N}\times\nabla\times\mathbf{N}\right)^{2}\\
 & +\frac{\tau_{1}}{2}P_{0}^{2}\mathbf{N}^{2}+\frac{\tau_{2}}{2}P_{0}^{4}\mathbf{N}^{4}+\frac{h}{2}P_{0}^{2}\frac{1}{\xi^{2}}\left(\nabla\mathbf{N}\right)^{2}-\gamma P_{0}\frac{1}{\xi}\left(\nabla\cdot\mathbf{N}\right)\mathbf{N}\cdot\mathbf{N}.
\end{align*}

\begin{align*}
\int_{\Omega}f= & \xi^{2}\left(\frac{K_{11}}{2}\frac{1}{\xi^{2}}\left(\nabla\cdot\mathbf{N}\right)^{2}+\frac{K_{22}}{2}\frac{1}{\xi^{2}}\left(\mathbf{N}\cdot\nabla\times\mathbf{N}\right)^{2}+\frac{K_{33}}{2}\frac{1}{\xi^{2}}\left(\mathbf{N}\times\nabla\times\mathbf{N}\right)^{2}\right)\\
 & +\xi^{2}\left(\frac{\tau_{1}}{2}P_{0}^{2}\mathbf{N}^{2}+\frac{\tau_{2}}{2}P_{0}^{4}\mathbf{N}^{4}+\frac{h}{2}P_{0}^{2}\frac{1}{\xi^{2}}\left(\nabla\mathbf{N}\right)^{2}-\gamma P_{0}\frac{1}{\xi}\left(\nabla\cdot\mathbf{N}\right)\mathbf{N}\cdot\mathbf{N}\right).
\end{align*}
\begin{align*}
\int_{\Omega}f= & \frac{K_{11}}{2}\left(\nabla\cdot\mathbf{N}\right)^{2}+\frac{K_{22}}{2}\left(\mathbf{N}\cdot\nabla\times\mathbf{N}\right)^{2}+\frac{K_{33}}{2}\left(\mathbf{N}\times\nabla\times\mathbf{N}\right)^{2}\\
 & +\frac{\tau_{1}}{2}P_{0}^{2}\xi^{2}\mathbf{N}^{2}+\frac{\tau_{2}}{2}P_{0}^{4}\xi^{2}\mathbf{N}^{4}+\frac{h}{2}P_{0}^{2}\left(\nabla\mathbf{N}\right)^{2}-\gamma P_{0}\xi\left(\nabla\cdot\mathbf{N}\right)\mathbf{N}\cdot\mathbf{N}.
\end{align*}


\subsection*{One Constant Approximation}

In the one constant approximation the Fredericks free energy reduces
to $\frac{1}{2}K_{1}\left(\nabla\mathbf{N}\right)^{2}$, thus it follows
that 
\[
f=\frac{K_{11}}{2}\left(\nabla\cdot\mathbf{N}\right)^{2}+\frac{\tau_{1}}{2}\mathbf{P}^{2}+\frac{\tau_{2}}{2}\mathbf{P}^{4}+\frac{h}{2}\left(\nabla\mathbf{P}\right)^{2}-\gamma\left(\nabla\cdot\mathbf{N}\right)\mathbf{N}\cdot\mathbf{P},
\]
again, without taking the depolarization effect. We could write the
above free energy in terms of just $\mathbf{N}$ as follows 
\[
f=\frac{K_{11}}{2}\left(\nabla\cdot\mathbf{N}\right)^{2}+\frac{\tau_{1}P_{0}^{2}}{2}\mathbf{N}^{2}+\frac{\tau_{2}P_{0}^{4}}{2}\mathbf{N}^{4}+\frac{hP_{0}^{2}}{2}\left(\nabla\mathbf{N}\right)^{2}-\gamma P_{0}\left(\nabla\cdot\mathbf{N}\right)\mathbf{N}\cdot\mathbf{N},
\]
Now, let's choose $K_{11}$ for rescaling and $\xi$ for rescaling
and dimensionless purposes respectively 
\[
f^{\prime}=\frac{1}{2}\left(\nabla\cdot\mathbf{N}\right)^{2}+\frac{1}{2}\tau_{1}^{\prime}\mathbf{N}^{2}+\frac{1}{2}\tau_{2}^{\prime}\mathbf{N}^{4}+\frac{1}{2}h^{\prime}\left(\nabla\mathbf{N}\right)^{2}-\gamma^{\prime}\left(\nabla\cdot\mathbf{N}\right)\mathbf{N}\cdot\mathbf{N},
\]
where the constants are given by 

\[
\tau_{1}^{\prime}=\frac{\tau_{1}P_{0}^{2}}{K_{11}}\xi^{2},\tau_{2}^{\prime}=\frac{\tau_{2}P_{0}^{4}}{K_{11}}\xi^{2},h^{\prime}=\frac{hP_{0}^{2}}{K_{11}},\gamma^{\prime}=\frac{\gamma P_{0}}{K_{11}}\xi.
\]
The numerical values of the constants are given by

\begin{minipage}[t]{1\columnwidth}%
\begin{center}
\begin{tabular}{|c|c|c|}
\hline 
Parameters & Values & Rescaled Values\tabularnewline
\hline 
\hline 
$K_{11}$ & $10^{-12}N$ & \tabularnewline
\hline 
$K_{33}$ & $1.2\times10^{-12}N$ & \tabularnewline
\hline 
$P_{0}$ & $0.045\text{\ensuremath{\frac{C}{m^{2}}}}$ & \tabularnewline
\hline 
$\tau_{1}$ & $-10^{3}\frac{Jm}{C}$ & \tabularnewline
\hline 
$\tau_{2}$ & $9.88\times10^{5}\frac{Jm^{5}}{C^{4}}$ & \tabularnewline
\hline 
$\gamma$ & $10^{-4}V$ & \tabularnewline
\hline 
$\varepsilon$ & $10^{-7}\frac{F}{m}$ & \tabularnewline
\hline 
$\beta$ & $10^{-5}$ & \tabularnewline
\hline 
$W_{Q}$ & $1\times10^{-6}\frac{J}{m^{2}}$ & \tabularnewline
\hline 
$W_{P}$ & $1\times10^{-6}\frac{J}{m^{2}}$ & \tabularnewline
\hline 
$h$ & $10^{-10}Jm^{3}C^{-2}$ & \tabularnewline
\hline 
\end{tabular}
\par\end{center}%
\end{minipage}

And from this we can compute the numerical values as follows
\begin{align*}
\tau_{1}^{\prime} & =-\frac{\left(10^{3}\right)\left(0.045\right)^{2}}{\left(10^{-12}\right)}\xi^{2},\\
\tau_{2}^{\prime} & =\frac{\left(9.88\times10^{5}\right)\left(0.045\right)^{4}}{\left(10^{-12}\right)}\xi^{2},\\
h^{\prime} & =\frac{\left(10^{-10}\right)\left(0.045\right)^{2}}{\left(10^{-12}\right)},\\
\gamma^{\prime} & =\frac{\left(10^{-4}\right)\left(0.045\right)}{\left(10^{-12}\right)}\xi
\end{align*}

therefore, we have 
\begin{align*}
\tau_{1}^{\prime} & =-2.025\times10^{12}\xi^{2}\\
\tau_{2}^{\prime} & =4.05142\times10^{11}\xi^{2}\\
h^{\prime} & =0.2025\\
\gamma^{\prime} & =4.5\times10^{6}\xi
\end{align*}
and if we choose $\xi=1\times10^{-6},$ then we have
\begin{align*}
\tau_{1}^{\prime} & =-2.025\\
\tau_{2}^{\prime} & =0.405142\\
h^{\prime} & =0.2025\\
\gamma^{\prime} & =4.5
\end{align*}
 sand for the surface terms we have $ $the following expressions
for the dimensionless parameters
\[
\omega_{Q}^{\prime}=\frac{W_{Q}}{K_{11}}\xi,\hspace{1em}\omega_{P}^{\prime}=\frac{W_{P}}{K_{11}}\xi
\]

\textbf{It makes sense to consider $\xi$ as the size of the domain.}

For the CV type here are some numerical values for the phenomenological
parameters;
\[
K_{11}=10^{-12}N,
\]

\textbf{What about the Depolarization Effect?}

By definition, the depolarization effect occurs when the electric
field inside a material is reduced due to the alignment of dipoles
within the material, which produce their own opposing field. For the
CV-type we have the following expression for the depolarization

\[
f=\frac{1}{2}P_{\text{eff}}\cos\left(b+kr\right)\left[\frac{1}{r}+\frac{\beta P_{\text{eff}}\cos\left(b+kr\right)}{\epsilon}\right],
\]

Under the one constant approximation we have the following free energy

\[
f^{\prime}=K_{1}^{\prime}\left(\nabla\cdot\mathbf{N}\right)^{2}+\tau_{1}^{\prime}\mathbf{N}^{2}+\tau_{2}^{\prime}\mathbf{N}^{4}+h^{\prime}\left(\nabla\mathbf{N}\right)^{2}-\gamma^{\prime}\left(\nabla\cdot\mathbf{N}\right)\mathbf{N}\cdot\mathbf{N},
\]
with the same dimensionless constants as in the full model.

\textbf{Question: Is there a special boundary condition for the electric
field---already know that P is tangent to the boundary.}

Results So Far:

Landau:

To-Do: $+\frac{\tau_{1}}{2}P_{0}^{2}\mathbf{N}^{2}+\frac{\tau_{2}}{2}P_{0}^{4}\mathbf{N}^{4}$
\begin{enumerate}
\item \textbf{Landau only (Done)}
\item \textbf{Landau + $|\nabla P|^{2}$ : We observe some kind of.}
\item \textbf{Landau + $|\nabla P|^{2}$ + Anchoring: We observe some kind
of structural transition.}
\item Landau + $|\nabla P|^{2}$+Frank
\item Everything
\item Learn that solving Poisson is equivalent to minimizing $\int\left(\left|\nabla\phi\right|^{2}-\rho\phi\right)d^{n}x$
\end{enumerate}

\subsection{Code Section}


\begin{lstlisting}
\end{lstlisting}

To-Do:

Use Mathematica to work out the equilibrium value of P as a function
of T just for the Landau Expansion (using dimensionless parameters).
\begin{thebibliography}{1}
\bibitem{key-1}Rosseto, Michely P., and Jonathan V. Selinger. \textquotedbl Theory
of the splay nematic phase: single versus double splay.\textquotedbl{}
Physical Review E 101, no. 5 (2020): 052707.

\end{thebibliography}

\end{document}
